\documentclass[12pt, a4paper]{article}
\usepackage[utf8]{inputenc}
\usepackage{proof}
\usepackage{framed}
\usepackage[spanish]{babel}
\usepackage[pdftex]{graphicx}
\usepackage{amsmath,amsthm,amssymb,amsfonts,latexsym,cancel}
\usepackage{float}
\usepackage{hyperref}
\usepackage{stmaryrd}
\usepackage{blindtext, rotating}
\usepackage{geometry}
\usepackage{multicol, multirow}
\usepackage{fancyhdr}
\usepackage{fancyvrb}
\usepackage{verbatim} %sacar si no usamos verbatim!!
\usepackage{listings} % Required for insertion of code
\usepackage[usenames,dvipsnames]{color} % Required for custom colors
\usepackage{anysize}
%\everymath{\displaystyle}

\setlength{\parindent}{0cm}
\setlength{\parskip}{\baselineskip}

\definecolor{orange}{RGB}{255,127,0}

%------------------------------CONFIGURACIÓN CÓDIGO---------------------------------------------
\definecolor{MyDarkGreen}{rgb}{0.0,0.4,0.0} % This is the color used for comments

\definecolor{identifierColor}{rgb}{0.65,0.16,0.16}

\lstset{language=Haskell,
        frame=single, % Single frame around code
        basicstyle=\small\ttfamily, % Use small true type font
        keywordstyle=[1]\color{Blue}\bf, % Perl functions bold and blue
        keywordstyle=[2]\color{Black}, % Perl function arguments purple
        keywordstyle=[3]\color{Blue}\underbar, % Custom functions underlined and blue
        identifierstyle=\color{identifierColor}, % Nothing special about identifiers                                         
        commentstyle=\usefont{T1}{pcr}{m}{sl}\color{MyDarkGreen}\small, % Comments small dark green courier font
        stringstyle=\color{Black}, % Strings are purple
        showstringspaces=false, % Don't put marks in string spaces
        tabsize=5, % 5 spaces per tab
        %
        % Put standard Perl functions not included in the default language here
        morekeywords={rand},
        %
        % Put Perl function parameters here
        morekeywords=[2]{on, off, interp},
        %
        % Put user defined functions here
        morekeywords=[3]{test},
       	%
        morecomment=[l][\color{Blue}]{...}, % Line continuation (...) like blue comment
        numbers=left, % Line numbers on left
        firstnumber=1, % Line numbers start with line 1
        numberstyle=\tiny\color{Blue}, % Line numbers are blue and small
        stepnumber=5 % Line numbers go in steps of 5
}
%-----------------------------------------------------------------------------------------------


%-------------------------------PORTADA---------------------------------------
\newcommand{\hmwkClassTitle}{Trabajo Práctico III}
\newcommand{\hmwkClassName}{Análisis de Lenguajes de Programación}

\title{
\vspace{2in}
\textmd{\textbf{\hmwkClassTitle}}\\
%\normalsize\vspace{0.1in}\small{Entrega:\ \hmwkDueDate}\\
\vspace{0.1in}\large{\textit{\hmwkClassName}}
\vspace{3in}
}
\date{15 de octubre de 2015}
\author{Borrero, Paula (P-4415/6) \and Herranz, Cecilia (H-0471/5) }
%-----------------------------------------------------------------------------

\begin{document}
\maketitle
\marginsize{2cm}{2cm}{2cm}{2cm}
\newpage
{\bf Ejercicio 1}\\
\\
Árbol de derivación de tipos de $S = \lambda x:B\rightarrow B\rightarrow B.\lambda y:B\rightarrow B.\lambda z:B.(x z) (y z)$ en página 4.

El resultado del árbol queda comprobado usando el intérprete:
\vspace{-0.5cm}
\begin{verbatim}
ST> :type S
(B -> B -> B) -> (B -> B) -> B -> B
\end{verbatim}

{\bf Ejercicio 2}\\
\\
infer retorna un valor de tipo Either para poder devolver mensajes de error cuando no se pueda inferir el tipo de la expresión.

$\gg =$ toma un valor v de tipo Either String Type y una función f de tipo Type $\rightarrow$ Either String Type y retorna Left v si es una cadena (Left) o la función f aplicada a v si es un tipo (Right). Sirve para propagar errores.
\\

{\bf Ejercicio 5}\\
\\
Árbol de tipado para $(let\,\, z = ((\lambda x : B. x)\,\, as\,\, B \rightarrow B)\,\, in\,\,z)\,\, as\,\, B \rightarrow B$:
\begin{center}
\begin{minipage}{\linewidth}
\infer[\scriptstyle T-ASCRIBE]{\vdash \, \, (let\, z\, = ((\lambda x:B.x)\, \, as\, \, B \rightarrow B)\, \, in \, \, z)\,\, as \, \, B \rightarrow B: B\rightarrow B}{\infer[\scriptstyle T-LET]{\vdash \, \, (let \, z\, =  ((\lambda x:B.x) \, \, as \, \, B \rightarrow B) \, \, in \, \, z):B \rightarrow B}{
	\infer[\scriptstyle T-ASCRIBE]{\vdash \, \,(\lambda x:B.x) \, \, as \, \, B \rightarrow B:B \rightarrow B}{\infer[\scriptstyle T-ABS]{\vdash \, \,(\lambda x:B.x):B \rightarrow B}{\infer[\scriptstyle T-VAR]{x:B\vdash x:B}{x:B \in x:B}}}
	&
	\infer[\scriptstyle T-VAR]{z:B \rightarrow B \vdash z:B \rightarrow B}{z:B \rightarrow B \in z:B \rightarrow B}
}}
\end{minipage}
\end{center}

El resultado del árbol queda comprobado usando el intérprete:
\vspace{-0.5cm}
\begin{verbatim}
ST> :type (let z = ((\x:B.x) as B -> B) in z) as B->B
B -> B
\end{verbatim}

\newpage
{\bf Ejercicio 7}\\
\\
Reglas de evaluación para pares:\\

\begin{multicols}{2}
\infer[\scriptstyle (E-PAIR1)]{(t_{1},t_{2}) \rightarrow (t'_{1},t_{2})}{t_{1} \rightarrow t'_{1}}
\vspace{0.5cm}
\infer[\scriptstyle (E-FST0)]{FST\,\, t \rightarrow FST\, \, t'}{t \rightarrow t'}
\vspace{0.5cm}
\infer[\scriptstyle (E-FST1)]{FST\,\, (v_{1},v_{2}) \rightarrow v_{1}}{}
\columnbreak
\infer[\scriptstyle (E-PAIR2)]{(v,t_{2}) \rightarrow (v, t'_{2})}{t_{2} \rightarrow t'_{2}}
\vspace{0.5cm}
\infer[\scriptstyle (E-SND0)]{SND\,\, t \rightarrow SND\, \, t'}{t \rightarrow t'}
\vspace{0.5cm}
\infer[\scriptstyle (E-SND1)]{SND\,\, (v_{1},v_{2}) \rightarrow v_{2}}{}
\end{multicols}
{\bf Ejercicio 9}\\
\\
Árbol de tipado para $fst\, \, (unit\, \, as \, \, Unit,\, \lambda x : (B, B). snd\, \, x)$:
\\
\begin{center}
\begin{minipage}{\linewidth}
\infer[\scriptstyle T-FST]{\vdash fst\, \, (unit\, \, as\, \, Unit, \lambda x:(B,B).snd\, \, x): Unit}{\infer[\scriptstyle T-PAIR]{\vdash (unit\, \, as\, \, Unit, \lambda x:(B,B).snd\, \, x): (Unit, (B \rightarrow B) \rightarrow B)}{\infer[\scriptstyle T-ASCRIBE]{\vdash unit\, \, as\, \, Unit:Unit}{\infer[\scriptstyle T-UNIT]{\vdash unit:Unit}{}}
&
\infer[\scriptstyle T-ABS]{\vdash \lambda x:(B,B).snd\, \, x:(B\rightarrow B)\rightarrow B}{\infer[\scriptstyle T-SND]{x:(B,B) \vdash snd\, \, x:B}{\infer[\scriptstyle T-VAR]{x:(B,B) \vdash x:(B,B)}{x:(B,B)\in x:(B,B)}}}
}}
\end{minipage}
\end{center}
El resultado del árbol queda comprobado usando el intérprete:
\vspace{-0.5cm}
\begin{verbatim}
ST> :type fst (unit as Unit, \x:(B,B). snd x)
Unit
\end{verbatim}
{\bf Ejercicio 11}\\
\\
La función Ack queda definida como:
\vspace{-0.5cm}
\begin{verbatim}
def ack = \m:Nat. R (\n:Nat. suc n) (\x:Nat->Nat. \y:Nat. \n:Nat. R (x (suc 0)) 
                                                       (\p:Nat. \q:Nat. x p) n) m
\end{verbatim}
\newpage
\marginsize{2cm}{2cm}{2cm}{2cm}

\newgeometry{top=11cm}
\begin{center}
\begin{turn}{90}
\begin{minipage}{\linewidth}
\begin{multicols}{2}
\infer[\scriptstyle T-ABS]{\scriptstyle \vdash \backslash x :B\rightarrow B\rightarrow B. \backslash y:B\rightarrow B. \backslash z: B. (x z)(y z):(B \rightarrow B\rightarrow B)\rightarrow (B\rightarrow B)\rightarrow B\rightarrow B}{
\infer[\scriptstyle T-ABS]{\scriptstyle x:B\rightarrow B\rightarrow B\,\, \vdash \, \, \backslash y:B\rightarrow B.\backslash z:B.(x z)(y z):(B\rightarrow B)\rightarrow B\rightarrow B}{
\infer[\scriptstyle T-ABS]{\scriptstyle x:B\rightarrow B\rightarrow B,\, y:B\rightarrow B \, \, \vdash \, \, \backslash z:B.(x z)(y z):B\rightarrow B}{
\infer[\scriptstyle T-APP]{\scriptstyle x:B\rightarrow B\rightarrow B,\, y:B\rightarrow B, z:B \, \, \vdash \, \, (x z)(y z):B}{
	\infer[\scriptstyle T-APP]{\scriptstyle x:B\rightarrow B\rightarrow B, y:B\rightarrow B, z:B \, \, \vdash \,\, (x z):B \rightarrow B}{
	\infer[\scriptstyle T-VAR]{\scriptstyle x:B\rightarrow B\rightarrow B, y:B\rightarrow B, z:B \, \, \vdash \,\, x:B\rightarrow B\rightarrow B}{}
	&
	\infer[\scriptstyle T-VAR]{\scriptstyle x:B\rightarrow B\rightarrow B, y:B\rightarrow B, z:B \, \, \vdash \,\, z:B}{}
}
&
	\infer[\scriptstyle T-APP]{\scriptstyle x:B\rightarrow B\rightarrow B, y:B\rightarrow B, z:B \, \, \vdash \, \, (y z):B}{
	\infer[\scriptstyle T-VAR]{\scriptstyle x:B\rightarrow B\rightarrow B, y:B\rightarrow B, z:B \, \, \vdash \,\, y:B\rightarrow B}{}
	&
	\infer[\scriptstyle T-VAR]{\scriptstyle x:B\rightarrow B\rightarrow B, y:B\rightarrow B, z:B \, \, \vdash \,\, z:B}{}}}
}
}
}
\end{multicols}
\end{minipage}
\end{turn}
\end{center}
\newpage
\newgeometry{}
\marginsize{2cm}{2cm}{2cm}{2cm}
{\bf Los ejercicios 3, 4, 6, 8 y 10 se encuentran resueltos en el código escrito a continuación:}\\
\\
Código de Common.hs:
\vspace{-1cm}

\begin{Verbatim}[commandchars=\\\{\}]

\textcolor{blue}{module Common where}

  \textcolor{red}{-- Comandos interactivos o de archivos}
  \textcolor{blue}{data} \textcolor{orange}{Stmt} i = \textcolor{orange}{Def String} i   \textcolor{red}{--  Declarar un nuevo identificador x, let x = t}
              | \textcolor{orange}{Eval i}         \textcolor{red}{--  Evaluar el término}
    \textcolor{blue}{deriving} (\textcolor{orange}{Show})
  
  \textcolor{blue}{instance} \textcolor{orange}{Functor Stmt} \textcolor{blue}{where}
    fmap f (\textcolor{orange}{Def} s i) = \textcolor{orange}{Def} s (f i)
    fmap f (\textcolor{orange}{Eval} i)  = \textcolor{orange}{Eval} (f i)

  \textcolor{red}{-- Tipos de los nombres}
  \textcolor{blue}{data} \textcolor{orange}{Name}
     =  \textcolor{orange}{Global  String}
     |  \textcolor{orange}{Quote   Int}
    \textcolor{blue}{deriving} (\textcolor{orange}{Show, Eq})

  \textcolor{red}{-- Entornos}
  \textcolor{blue}{type} \textcolor{orange}{NameEnv} v t = [(\textcolor{orange}{Name}, (v, t))]

  \textcolor{red}{-- Tipo de los tipos}
  \textcolor{blue}{data} \textcolor{orange}{Type} = \textcolor{orange}{Base}
            | \textcolor{orange}{Unit}
            | \textcolor{orange}{Nat}
            | \textcolor{orange}{Fun Type Type}
            | \textcolor{orange}{Tup Type Type}
            \textcolor{blue}{deriving} (\textcolor{orange}{Show, Eq})
  
  \textcolor{red}{-- Términos con nombres}
  \textcolor{blue}{data} \textcolor{orange}{LamTerm}  =  \textcolor{orange}{LUnit}
                |  \textcolor{orange}{LZero}
                |  \textcolor{orange}{LSuc LamTerm}
                |  \textcolor{orange}{LR LamTerm LamTerm LamTerm}
                |  \textcolor{orange}{LVar String}
                |  \textcolor{orange}{Abs String Type LamTerm}
                |  \textcolor{orange}{App LamTerm LamTerm}
                |  \textcolor{orange}{LLet String LamTerm LamTerm}
                |  \textcolor{orange}{LAs LamTerm Type}
                |  \textcolor{orange}{LTup LamTerm LamTerm}
                |  \textcolor{orange}{LFst LamTerm}
                |  \textcolor{orange}{LSnd LamTerm}
                \textcolor{blue}{deriving} (\textcolor{orange}{Show, Eq})


  \textcolor{red}{-- Términos localmente sin nombres}
  \textcolor{blue}{data} \textcolor{orange}{Term}  = \textcolor{orange}{TUnit}
             | \textcolor{orange}{Zero}
             | \textcolor{orange}{Bound Int}
             | \textcolor{orange}{Free Name}
             | \textcolor{orange}{Suc Term}
             | \textcolor{orange}{R Term Term Term}
             | \textcolor{orange}{Term :@: Term}
             | \textcolor{orange}{Lam Type Term}
             | \textcolor{orange}{Let Term Term}
             | \textcolor{orange}{As Term Type}
             | \textcolor{orange}{TTup Term Term}
             | \textcolor{orange}{Fst Term}
             | \textcolor{orange}{Snd Term}
          \textcolor{blue}{deriving} (\textcolor{orange}{Show, Eq})

  \textcolor{red}{-- Valores}
  \textcolor{blue}{data} \textcolor{orange}{Value} = \textcolor{orange}{VUnit}
             | \textcolor{orange}{VZero}
             | \textcolor{orange}{VSuc Value}
             | \textcolor{orange}{VLam Type Term}
             | \textcolor{orange}{VTup Value Value}
            \textcolor{blue}{deriving} \textcolor{orange}{Show}

  \textcolor{red}{-- Contextos del tipado}
  \textcolor{blue}{type} \textcolor{orange}{Context} = [\textcolor{orange}{Type}]
\end{Verbatim}
Código de Symplytyped.hs
\vspace{-0.5cm}
\begin{Verbatim}[commandchars=\&\{\}]
&textcolor{blue}{module Simplytyped} (
       conversion,    &textcolor{red}{-- conversion a terminos localmente sin nombre}
       eval,          &textcolor{red}{-- evaluador}
       infer,         &textcolor{red}{-- inferidor de tipos}
       quote          &textcolor{red}{-- valores -> terminos}
       )
       &textcolor{blue}{where}

&textcolor{blue}{import Data.List}
&textcolor{blue}{import Data.Maybe}
&textcolor{blue}{import Prelude hiding ((>>=))}
&textcolor{blue}{import Text.PrettyPrint.HughesPJ (render)}
&textcolor{blue}{import PrettyPrinter}
&textcolor{blue}{import Common}

&textcolor{red}{-- conversion a términos localmente sin nombres}
conversion :: &textcolor{orange}{LamTerm} -> &textcolor{orange}{Term}
conversion = conversion' []

conversion' :: [&textcolor{orange}{String}] -> &textcolor{orange}{LamTerm} -> &textcolor{orange}{Term}
conversion' b &textcolor{orange}{LUnit}          = &textcolor{orange}{TUnit}
conversion' b &textcolor{orange}{LZero}          = &textcolor{orange}{Zero}
conversion' b (&textcolor{orange}{LVar} n)       = maybe (&textcolor{orange}{Free} (&textcolor{orange}{Global} n)) &textcolor{orange}{Bound} (n `elemIndex` b)
conversion' b (&textcolor{orange}{LFst} t)       = &textcolor{orange}{Fst} (conversion' b t)
conversion' b (&textcolor{orange}{LSnd} t)       = &textcolor{orange}{Snd} (conversion' b t)
conversion' b (&textcolor{orange}{App} t u)      = conversion' b t :@: conversion' b u
conversion' b (&textcolor{orange}{Abs} n t u)    = &textcolor{orange}{Lam} t (conversion' (n:b) u)
conversion' b (&textcolor{orange}{LLet} n t1 t2) = &textcolor{orange}{Let} (conversion' b t1) (conversion' (n:b) t2)
conversion' b (&textcolor{orange}{LAs} u t)      = &textcolor{orange}{As} (conversion' b u) t
conversion' b (&textcolor{orange}{LTup} t1 t2)   = &textcolor{orange}{TTup} (conversion' b t1) (conversion' b t2)
conversion' b (&textcolor{orange}{LSuc} t)       = &textcolor{orange}{Suc} (conversion' b t)
conversion' b (&textcolor{orange}{LR} t0 t1 t2)  = &textcolor{orange}{R} (conversion' b t0) (conversion' b t1)
                                                               (conversion' b t2)

 
&textcolor{red}{--- eval ----------------------}

sub :: &textcolor{orange}{Int} -> &textcolor{orange}{Term} -> &textcolor{orange}{Term} -> &textcolor{orange}{Term}
sub i t (&textcolor{orange}{Bound} j) | i == j    = t
sub _ _ (&textcolor{orange}{Bound} j) | otherwise = &textcolor{orange}{Bound} j
sub _ _ (&textcolor{orange}{Free} n)              = &textcolor{orange}{Free} n
sub _ _ &textcolor{orange}{TUnit}                 = &textcolor{orange}{TUnit}
sub _ _ &textcolor{orange}{Zero}                  = &textcolor{orange}{Zero}
sub i t (&textcolor{orange}{Fst} t')              = &textcolor{orange}{Fst} (sub i t t')
sub i t (&textcolor{orange}{Snd} t')              = &textcolor{orange}{Snd} (sub i t t')
sub i t (u :@: v)             = sub i t u :@: sub i t v
sub i t (&textcolor{orange}{Lam} t' u)            = &textcolor{orange}{Lam} t' (sub (i+1) t u)
sub i t (&textcolor{orange}{Let} t0 t1)           = &textcolor{orange}{Let} (sub i t t0) (sub (i+1) t t1)
sub i t (&textcolor{orange}{As} u t')             = &textcolor{orange}{As} (sub i t u) t'
sub i t (&textcolor{orange}{Suc} t')              = &textcolor{orange}{Suc} (sub i t t')
sub i t (&textcolor{orange}{TTup} t0 t1)          = &textcolor{orange}{TTup} (sub i t t0) (sub i t t1)
sub i t (&textcolor{orange}{R} t0 t1 t2)          = &textcolor{orange}{R} (sub i t t0) (sub i t t1) (sub i t t2)

&textcolor{red}{-- evaluador de términos}
eval :: &textcolor{orange}{NameEnv Value Type} -> &textcolor{orange}{Term} -> &textcolor{orange}{Value}
eval _ &textcolor{orange}{TUnit}                 = &textcolor{orange}{VUnit}
eval _ &textcolor{orange}{Zero}                  = &textcolor{orange}{VZero}
eval e (&textcolor{orange}{Suc} t)               = &textcolor{orange}{VSuc} (eval e t)
eval _ (&textcolor{orange}{Bound} _)             = error &textcolor{orange}{"variable ligada inesperada en eval"}
eval e (&textcolor{orange}{Free} n)              = fst $ fromJust $ lookup n e
eval _ (&textcolor{orange}{Lam} t u)             = &textcolor{orange}{VLam} t u
eval e (&textcolor{orange}{Lam} _ u :@: Lam s v) = eval e (sub 0 (&textcolor{orange}{Lam} s v) u)
eval e (&textcolor{orange}{Lam} t u :@: v)       = &textcolor{blue}{case} eval e v &textcolor{blue}{of}
                 &textcolor{orange}{VLam} t' u' -> eval e (&textcolor{orange}{Lam} t u :@: &textcolor{orange}{Lam} t' u')
                 &textcolor{orange}{VUnit}      -> eval e (sub 0 &textcolor{orange}{TUnit} u)
                 &textcolor{orange}{VZero}      -> eval e (sub 0 &textcolor{orange}{Zero} u)
                 &textcolor{orange}{VSuc} x     -> eval e (sub 0 (quote (V&textcolor{orange}{Suc} x)) u)
                 _          -> error &textcolor{orange}{"Error de tipo en run-time,}
                                                  &textcolor{orange}{verificar type checker"}
eval e (u :@: v)             = &textcolor{blue}{case} eval e u &textcolor{blue}{of}
                 &textcolor{orange}{VLam} t u' -> eval e (&textcolor{orange}{Lam} t u' :@: v)
                 _         -> error &textcolor{orange}{"Error de tipo en run-time,}
                                                   &textcolor{orange}{verificar type checker"}
eval e (&textcolor{orange}{Let} t0 t1)           = eval e (sub 0 (quote (eval e t0)) t1)
eval e (&textcolor{orange}{As} u t)              = eval e u
eval e (&textcolor{orange}{Fst} t)               = &textcolor{blue}{case} eval e t &textcolor{blue}{of}
                                    &textcolor{orange}{VTup} t0 t1 -> t0
                                    _          -> error &textcolor{orange}{"Error de tipo en run-time,}
                                                   &textcolor{orange}{el argumento debe ser una tupla"}
eval e (&textcolor{orange}{Snd} t)               = &textcolor{blue}{case} eval e t &textcolor{blue}{of}
                                    &textcolor{orange}{VTup} t0 t1 -> t1
                                    _          -> error &textcolor{orange}{"Error de tipo en run-time,}
                                                  &textcolor{orange}{el argumento debe ser una tupla"}
eval e (&textcolor{orange}{TTup} t0 t1)          = &textcolor{orange}{VTup} (eval e t0) (eval e t1)
eval e (&textcolor{orange}{R} t0 t1 t2) = &textcolor{blue}{case} eval e t2 &textcolor{blue}{of}
                       &textcolor{orange}{VZero}  -> eval e t0
                       &textcolor{orange}{VSuc} t -> eval e ((t1 :@: (&textcolor{orange}{R} t0 t1 (quote t))) :@: (quote t))
	              _      -> error &textcolor{orange}{"Error de tipo en run-time, el} 
	                                  &textcolor{orange}{tercer argumento de R tiene que ser Nat"}


&textcolor{red}{-----------------------}
&textcolor{red}{------ quoting}
&textcolor{red}{-----------------------}

quote :: &textcolor{orange}{Value} -> &textcolor{orange}{Term}
quote &textcolor{orange}{VUnit}      = &textcolor{orange}{TUnit}
quote &textcolor{orange}{VZero} = &textcolor{orange}{Zero}
quote (&textcolor{orange}{VSuc} t) = &textcolor{orange}{Suc} (quote t)
quote (&textcolor{orange}{VLam} t f) = &textcolor{orange}{Lam} t f
quote (&textcolor{orange}{VTup} t0 t1) = &textcolor{orange}{TTup} (quote t0) (quote t1)

&textcolor{red}{-----------------------}
&textcolor{red}{--- type checker}
&textcolor{red}{-----------------------}

&textcolor{red}{-- type checker}
infer :: &textcolor{orange}{NameEnv Value Type} -> &textcolor{orange}{Term} -> &textcolor{orange}{Either String Type}
infer = infer' []

&textcolor{red}{-- definiciones auxiliares}
ret :: &textcolor{orange}{Type} -> &textcolor{orange}{Either String Type}
ret = &textcolor{orange}{Right}

err :: &textcolor{orange}{String} -> &textcolor{orange}{Either String Type}
err = &textcolor{orange}{Left}

(>>=) :: &textcolor{orange}{Either String Type} -> (&textcolor{orange}{Type} -> &textcolor{orange}{Either String Type}) -> &textcolor{orange}{Either String Type}
(>>=) v f = either &textcolor{orange}{Left} f v

&textcolor{red}{-- fcs. de error}

matchError :: &textcolor{orange}{Type} -> &textcolor{orange}{Type} -> &textcolor{orange}{Either String Type}
matchError t1 t2 = err $ &textcolor{orange}{"se esperaba "} ++
                         render (printType t1) ++
                         &textcolor{orange}{", pero "} ++
                         render (printType t2) ++
                         &textcolor{orange}{" fue inferido."}

notfunError :: &textcolor{orange}{Type} -> &textcolor{orange}{Either String Type}
notfunError t1 = err $ render (printType t1) ++ &textcolor{orange}{" no puede ser aplicado."}

notfoundError :: &textcolor{orange}{Name} -> &textcolor{orange}{Either String Type}
notfoundError n = err $ show n ++ &textcolor{orange}{" no está definida."}

infer' :: &textcolor{orange}{Context} -> &textcolor{orange}{NameEnv Value Type} -> &textcolor{orange}{Term} -> &textcolor{orange}{Either String Type}
infer' _ _ &textcolor{orange}{TUnit}     = ret &textcolor{orange}{Unit}
infer' _ _ &textcolor{orange}{Zero}      = ret &textcolor{orange}{Nat}
infer' c e (&textcolor{orange}{Suc} t)  = infer' c e t >>= \tt->
                            &textcolor{blue}{case} tt &textcolor{blue}{of}
                                &textcolor{orange}{Nat} -> ret &textcolor{orange}{Nat}
                                _   -> matchError &textcolor{orange}{Nat} tt
infer' c _ (&textcolor{orange}{Bound} i) = ret (c !! i)
infer' _ e (&textcolor{orange}{Free} n) = &textcolor{blue}{case} lookup n e &textcolor{blue}{of}
                        &textcolor{orange}{Nothing} -> notfoundError n
                        &textcolor{orange}{Just} (_,t) -> ret t
infer' c e (t :@: u) = infer' c e t >>= \tt -> 
                       infer' c e u >>= \tu ->
                       &textcolor{blue}{case} tt &textcolor{blue}{of}
                         &textcolor{orange}{Fun} t1 t2 -> &textcolor{blue}{if} (tu == t1) 
                                        &textcolor{blue}{then} ret t2
                                        &textcolor{blue}{else} matchError t1 tu
                         _         -> notfunError tt
infer' c e (&textcolor{orange}{Lam} t u) = infer' (t:c) e u >>= \tu ->
                       ret $ &textcolor{orange}{Fun} t tu
infer' c e (&textcolor{orange}{Let} t0 t1) = infer' c e t0 >>= \tt0 -> 
                                infer' (tt0:c) e t1
infer' c e (&textcolor{orange}{As} u t) = infer' c e u >>= \tu ->
                            &textcolor{blue}{if} tu == t &textcolor{blue}{then} ret t &textcolor{blue}{else} matchError t tu
infer' c e (&textcolor{orange}{TTup} t0 t1) = infer' c e t0 >>= \tt0 ->
                          infer' c e t1 >>= \tt1 ->
                               ret (&textcolor{orange}{Tup} tt0 tt1)
infer' c e (&textcolor{orange}{Fst} t) = infer' c e t >>= \tt ->
                        &textcolor{blue}{case} tt &textcolor{blue}{of}
                            &textcolor{orange}{Tup} tt0 tt1 -> ret tt0
                            _           -> notfunError tt
infer' c e (&textcolor{orange}{Snd} t) = infer' c e t >>= \tt ->
                        &textcolor{blue}{case} tt &textcolor{blue}{of}
                            &textcolor{orange}{Tup} tt0 tt1 -> ret tt1
                            _           -> notfunError tt
infer' c e (&textcolor{orange}{R} t0 t1 t2) = infer' c e t0 >>= \tt0 -> 
                          infer' c e t1 >>= \tt1 ->
                          infer' c e t2 >>= \tt2 ->
                          &textcolor{blue}{case} tt1 &textcolor{blue}{of}
                            &textcolor{orange}{Fun} t t' -> if t == tt0 &&&& t' == &textcolor{orange}{Fun Nat} t &&&& tt2 == &textcolor{orange}{Nat}
                                        &textcolor{blue}{then} ret t
                                        &textcolor{blue}{else} matchError (&textcolor{orange}{Fun} t t') tt1
                            _        -> notfunError tt1
\end{Verbatim}
Código de Parse.y:
\begin{Verbatim}
{
module Parse where
import Common
import Data.Maybe
import Data.Char

}

%monad { P } { thenP } { returnP }
%name parseStmt Def
%name parseStmts Defs
%name term Exp

%tokentype { Token }
%lexer {lexer} {TEOF}

%token
    '='     { TEquals }
    ':'     { TColon }
    '\\'    { TAbs }
    '.'     { TDot }
    '('     { TOpen }
    ')'     { TClose }
    '->'    { TArrow }
    ','     { TComma }
    LET     { TLet }
    IN      { TIn }
    VAR     { TVar $$ }
    TYPE    { TType }
    DEF     { TDef }
    TYPEUNIT { TTypeUnit }
    NAT     { TNat }
    ZERO    { TZero }
    SUC     { TSuc }
    REC     { TR }
    AS      { TAs }
    UNIT    { TokenUnit }
    FST     { TFst }
    SND     { TSnd }
    

%right VAR
%left '=' 
%right '->'
%right '\\' '.' LET IN
%left AS 
%right REC
%right SUC 
%right SND FST


%%

Def     :  Defexp                      { $1 }
        |  Exp	                       { Eval $1 }
Defexp  : DEF VAR '=' Exp              { Def $2 $4 } 

Exp     :: { LamTerm }
        : '\\' VAR ':' Type '.' Exp    { Abs $2 $4 $6 }
        | LET VAR '=' Exp IN Exp       { LLet $2 $4 $6 }
        | Exp AS Type                  { LAs $1 $3 }
        | REC Atom Atom Exp            { LR $2 $3 $4 }
        | FST Atom                     { LFst $2 }
        | SND Atom                     { LSnd $2 }
        | NAbs                         { $1 }
        
NAbs    :: { LamTerm }
        : NAbs Atom                    { App $1 $2 }
        | Atom                         { $1 }

Atom    :: { LamTerm }
        : VAR                          { LVar $1 }
        | SUC Atom                     { LSuc $2 }
        | '(' Exp ',' Exp ')'          { LTup $2 $4 }
        | '(' Exp ')'                  { $2 }
        | UNIT                         { LUnit }
        | ZERO                         { LZero }

Type    : TYPE                         { Base }
        | TYPEUNIT                     { Unit }
        | NAT                          { Nat } 
        | '(' Type ',' Type ')'        { Tup $2 $4 }
        | Type '->' Type               { Fun $1 $3 }
        | '(' Type ')'                 { $2 }

Defs    : Defexp Defs                  { $1 : $2 }
        |                              { [] }
     
{

data ParseResult a = Ok a | Failed String
                     deriving Show                     
type LineNumber = Int
type P a = String -> LineNumber -> ParseResult a

getLineNo :: P LineNumber
getLineNo = \s l -> Ok l

thenP :: P a -> (a -> P b) -> P b
m `thenP` k = \s l-> case m s l of
                         Ok a     -> k a s l
                         Failed e -> Failed e
                         
returnP :: a -> P a
returnP a = \s l-> Ok a

failP :: String -> P a
failP err = \s l -> Failed err

catchP :: P a -> (String -> P a) -> P a
catchP m k = \s l -> case m s l of
                        Ok a     -> Ok a
                        Failed e -> k e s l

happyError :: P a
happyError = \ s i -> Failed $ "Línea "++(show (i::LineNumber))++":
					                  Error de parseo\n"++(s)

data Token = TVar String
               | TType
               | TDef
               | TTypeUnit
               | TNat
               | TAbs
               | TDot
               | TOpen
               | TClose 
               | TColon
               | TArrow
               | TComma
               | TEquals
               | TLet
               | TAs
               | TIn
               | TokenUnit
               | TFst
               | TSnd
               | TZero
               | TSuc
               | TR
               | TEOF
               deriving Show

lexer cont s = case s of
                    [] -> cont TEOF []
                    ('\n':s)  ->  \line -> lexer cont s (line + 1)
                    (c:cs)
                          | isSpace c -> lexer cont cs
                          | isAlpha c -> lexVar (c:cs)
                    ('-':('-':cs)) -> lexer cont $ dropWhile ((/=) '\n') cs
                    ('{':('-':cs)) -> consumirBK 0 0 cont cs	
	            ('-':('}':cs)) -> \ line -> Failed $ "Línea "++(show line)++":
	            				        Comentario no abierto"
                    ('-':('>':cs)) -> cont TArrow cs
                    ('\\':cs)-> cont TAbs cs
                    ('.':cs) -> cont TDot cs
                    ('(':cs) -> cont TOpen cs
                    ('-':('>':cs)) -> cont TArrow cs
                    (')':cs) -> cont TClose cs
                    (':':cs) -> cont TColon cs
                    ('=':cs) -> cont TEquals cs
                    (',':cs) -> cont TComma cs
                    ('0':cs) -> cont TZero cs
                    unknown -> \line -> Failed $ "Línea "++(show line)++":
                          No se puede reconocer "++(show $ take 10 unknown)++ "..."
                    where lexVar cs = case span isAlpha cs of
                                           ("B",rest)   -> cont TType rest
                                           ("Unit", rest) -> cont TTypeUnit rest
                                           ("Nat", rest) -> cont TNat rest
                                           ("def",rest) -> cont TDef rest
                                           ("let",rest) -> cont TLet rest
                                           ("in",rest) -> cont TIn rest
                                           ("as", rest) -> cont TAs rest
                                           ("unit", rest) -> cont TokenUnit rest
                                           ("fst", rest) -> cont TFst rest
                                           ("snd", rest) -> cont TSnd rest
                                           ("suc", rest) -> cont TSuc rest
                                           ("R", rest) -> cont TR rest
                                           (var,rest)   -> cont (TVar var) rest
                          consumirBK anidado cl cont s = case s of
                                       		   ('-':('-':cs)) ->
                                       		      consumirBK anidado cl
                                       		      cont $ dropWhile
                                       		      ((/=) '\n') cs
		                                           ('{':('-':cs)) ->
		                                             consumirBK (anidado+1)
		                                             cl cont cs	
		                                             ('-':('}':cs)) ->
		                                              case anidado of
			                                       0 -> \line -> lexer
			                                            cont cs (line+cl)
			                                       _ -> consumirBK
			                                            (anidado-1) cl
			                                            cont cs
		                                              ('\n':cs) ->
		                                                consumirBK anidado
		                                                (cl+1) cont cs
		                                              (_:cs) -> consumirBK
		                                                 anidado cl cont cs     
                                           
stmts_parse s = parseStmts s 1
stmt_parse s = parseStmt s 1
term_parse s = term s 1
}
\end{Verbatim}
Código de PrettyPrinter.hs:
\begin{Verbatim}[commandchars=\&\{\}]
&textcolor{blue}{module PrettyPrinter} (
       printTerm,     &textcolor{red}{-- pretty printer para terminos}
       printType,     &textcolor{red}{-- pretty printer para tipos}
       )
       &textcolor{blue}{where}

&textcolor{orange}{import Common}
&textcolor{orange}{import Text.PrettyPrint.HughesPJ}

&textcolor{red}{-- lista de posibles nombres para variables}
vars :: [&textcolor{orange}{String}]
vars = [ c : n | n <- &textcolor{orange}{""} : map show [1..], c <- [&textcolor{orange}{'x'},&textcolor{orange}{'y'},&textcolor{orange}{'z'}] ++ [&textcolor{orange}{'a'}..&textcolor{orange}{'w'}] ]
              
parensIf :: &textcolor{orange}{Bool} -> &textcolor{orange}{Doc} -> &textcolor{orange}{Doc}
parensIf &textcolor{orange}{True}  = parens
parensIf &textcolor{orange}{False} = id

&textcolor{red}{-- pretty-printer de términos}

pp :: &textcolor{orange}{Int} -> [&textcolor{orange}{String}] -> &textcolor{orange}{Term} -> &textcolor{orange}{Doc}
pp ii vs &textcolor{orange}{TUnit}             = text &textcolor{orange}{"unit"}
pp ii vs &textcolor{orange}{Zero}              = text &textcolor{orange}{"0"}
pp ii vs (&textcolor{orange}{Suc} t)           = text &textcolor{orange}{"suc "} <>
                             parensIf (not (isAtm t)) (pp ii vs t)
pp ii vs (&textcolor{orange}{Bound} k)         = text (vs !! (ii - k - &textcolor{green}{1}))
pp _  vs (&textcolor{orange}{Free} (&textcolor{orange}{Global} s)) = text s
pp ii vs (i :@: c)         = sep [parensIf (isLam i) (pp ii vs i), 
                               nest &textcolor{green}{1} (parensIf (isLam c || isApp c) (pp ii vs c))]  
pp ii vs (&textcolor{orange}{Lam} t c)         = text &textcolor{orange}{"\\"} <>
                             text (vs !! ii) <>
                             text &textcolor{orange}{":"} <>
                             printType t <>
                             text &textcolor{orange}{". "} <> 
                             pp (ii+&textcolor{green}{1}) vs c
pp ii vs (&textcolor{orange}{Let} t0 t1)       = text &textcolor{orange}{"let "} <>
                             text (vs !! ii) <>
                             text &textcolor{orange}{" = "} <>
                             pp ii vs t0 <>
                             text &textcolor{orange}{" in "} <>
                             pp (ii+1) vs t1
pp ii vs (&textcolor{orange}{As} u t)          = pp ii vs u <>
                             text &textcolor{orange}{" as "} <>
                             printType t
pp ii vs (&textcolor{orange}{TTup} t0 t1)      = text &textcolor{orange}{"("} <>
                             pp ii vs t0 <>
                             text &textcolor{orange}{","} <>
                             pp ii vs t1 <>
                             text &textcolor{orange}{")"}
pp ii vs (&textcolor{orange}{Fst} t)           = text &textcolor{orange}{"fst "} <>
                             pp ii vs t
pp ii vs (&textcolor{orange}{Snd} t)           = text &textcolor{orange}{"snd"} <>
                             pp ii vs t
pp ii vs (&textcolor{orange}{R} t0 t1 t2)      = text &textcolor{orange}{"R"} <>
                             pp ii vs t0 <>
                             pp ii vs t1 <>
                             pp ii vs t2
                                                         
isLam (&textcolor{orange}{Lam} _ _) = &textcolor{orange}{True}
isLam  _      = &textcolor{orange}{False}
   
isApp (_ :@: _) = &textcolor{orange}{True}
isApp _         = &textcolor{orange}{False}

isAtm &textcolor{orange}{TUnit}        = &textcolor{orange}{True}
isAtm &textcolor{orange}{Zero}         = &textcolor{orange}{True}
isAtm (&textcolor{orange}{TTup} t0 t1) = &textcolor{orange}{True}
isAtm _            = &textcolor{orange}{False}



&textcolor{red}{-- pretty-printer de tipos}
printType :: &textcolor{orange}{Type} -> &textcolor{orange}{Doc}
printType &textcolor{orange}{Base}         = text &textcolor{orange}{"B"}
printType &textcolor{orange}{Unit}         = text &textcolor{orange}{"Unit"}
printType &textcolor{orange}{Nat}          = text &textcolor{orange}{"Nat"}
printType (&textcolor{orange}{Fun} t1 t2)  = sep [ parensIf (isFun t1) (printType t1), 
                               text &textcolor{orange}{"->"}, 
                               printType t2]
printType (&textcolor{orange}{Tup} t1 t2)  = text &textcolor{orange}{"("} <>
                         printType t1 <>
                         text &textcolor{orange}{","} <>
                         printType t2 <>
                         text &textcolor{orange}{")"}


isFun (&textcolor{orange}{Fun} _ _)        = &textcolor{orange}{True}
isFun _                = &textcolor{orange}{False}

fv :: &textcolor{orange}{Term} -> [&textcolor{orange}{String}]
fv &textcolor{orange}{TUnit}             = []
fv &textcolor{orange}{Zero}              = []
fv (&textcolor{orange}{Suc} t)           = fv t
fv (&textcolor{orange}{Bound} _)         = []
fv (&textcolor{orange}{Free} (&textcolor{orange}{Global} n)) = [n]
fv (&textcolor{orange}{Free} _)          = []
fv (t :@: u)         = fv t ++ fv u
fv (&textcolor{orange}{Lam} _ u)         = fv u
fv (&textcolor{orange}{Let} t0 t1)       = fv t0 ++ fv t1
fv (&textcolor{orange}{As} u t)          = fv u
fv (&textcolor{orange}{Fst} t)           = fv t
fv (&textcolor{orange}{Snd} t)           = fv t
fv (&textcolor{orange}{TTup} t0 t1)      = fv t0 ++ fv t1
fv (&textcolor{orange}{R} t0 t1 t2)      = fv t0 ++ fv t1 ++ fv t2
  
---
printTerm :: &textcolor{orange}{Term} -> &textcolor{orange}{Doc} 
printTerm t = pp &textcolor{green}{0} (filter (\v -> not $ elem v (fv t)) vars) t
\end{Verbatim}
\newpage
\end{document}
